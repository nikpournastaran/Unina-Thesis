\newpage\pagestyle{introduction}

% The \unnumberedchapter, \unnumberedsection commands provided by the uninathesis 
% documentclass add an unnumbered entry to the TOC.
% If you use the starred version of classic sectioning commands (e.g. \chapter*)
% no TOC entry will be added!

\unnumberedchapter{Introduction}
\unnumberedsection{A first unnumbered section}

There are many subjects related to logical reason
ing, including philosophy, logic, and AI. Among
 them, the definition and categorization aspects of
 logical reasoning are handled by philosophy re
search. However, debate exists in philosophy re
search on the categorization of logical reasoning.
 We leave a detailed description of the debate in
 philosophy research in §A.2 and only leave the
 conclusions here according to philosophy research.
 In general, logical reasoning consists of deduc
tive, inductive, and abductive reasoning (Console
 and Saitta, 2000). Given an argument consisting of
 premises and a conclusion, we define the sub-type
 of logical reasoning it involves below:
 Definition for deductive reasoning: the premises
 can conclusively provide support for the conclu
sion, i.e. if the premises are all true, it would be
 impossible for the conclusion to be false.
 Definition for inductive reasoning: the premises
 cannot conclusively provide support for the con
clusion, since the conclusion generalizes existing
 information in premises to new knowledge, which
 has a wider applicable scope than those in premises.
 Definition for abductive reasoning: the premises
 cannot conclusively provide support for the conclu
sion, since the conclusion contains more specific
 information over the premises (most commonly
 used as generating most probable explanations).
 Please note that according to Console and Saitta
 (2000), inductive reasoning and abductive reason
ing are not exclusive to each other.


\unnumberedsection{Advantages over Formal Language}
 Building and reasoning over formal language have
 proved challenging (Musen and Van der Lei, 1988;
 Cropper et al., 2022), with disadvantages such as
 (1) brittleness (expert system fails when its knowl
edge base does not contain complete knowledge
 for a problem), (2) knowledge-acquisition bottle
neck (human experts are needed to encode their
 knowledge with formal representation), (3) inabil
ity to handle raw data such as natural language,
 (4) sensitivity to label errors, and (5) failure to
 recognize different symbols with similar meanings.
 Nevertheless, the new paradigm of logical reasoning, LRNLP, has systematic strengths over these
challenges.


\unnumberedsection{Advantages over E2E Neural Methods}
As a neuro-symbolic method, LRNLP systemati
cally has some advantages over end-to-end neural
 methods, suchas interpretability Cambriaetal.,
 2023 (since it isususallystepwise),morecon
trollability (LRNLP reasons following a given
 knowledge base), and less catastrophic forget
ting(LRNLP uses an explicit knowledge base)


 AdvantagesoverNeuro-symbolicSystems
 LRNLPcouldbeseenasanewtypeofneuro
symbolicinadditiontotheexisting6types(Kautz,
 2022),asitsgoalanddesignofmethodologyare
 typicallysymbolic(logicalreasoningwithknowl
edgebases),whileavoidinganysymbolicrepre
sentation,using(currentlypure)neuralmethods.
 ThereforeLRNLPcanavoidmanybottlenecksof
 theotherneuro-symbolicmethodscausedbysym
bolicrepresentation,suchassymbolicknowledge
 acquisitionandscalability




\blindtext[3]

\unnumberedsection{Another unnumbered section}

\blindtext[3]

\unnumberedsection{About this thesis work}

Thesis structure, etc..
